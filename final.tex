\documentclass[report]{jlreq}

\usepackage{listings}

\author{2013553 中野将生}
\date{\today}
\title{S8 最終レポート}
\begin{document}
  \maketitle
  \chapter{成果物の概要}
    本実験で実装した処理系は\texttt{./code/bin/main.ml}と\texttt{./compiler/main.ml}の2種類である。
    前者の処理系は課題6-2までの機能を実装した木構造に対するインタプリタ、
    CAM、ZAMへのコンパイルと実行が可能である。第一引数に\texttt{tree}、\texttt{cam}、\texttt{zam}を指定し、
    第2引数にファイルを指定する事で実行できる。
    後者の処理系はパターンマッチ、高階型を含む多相型推論と、簡単なサブセット言語のコンパイル機能がある。

    \texttt{./code/bin/main.ml}は\texttt{code}ディレクトリで\texttt{dune build}を行うと\texttt{code/\_build/default/bin/main.exe}
    に実行可能ファイルが生成される。\texttt{./compiler/main.ml}は\texttt{./compiler}ディレクトリで\texttt{make}を実行すれば
    単体テストが実行されコンパイラがビルドされる。
  \chapter{実行例}
  \chapter{実装したコンパイラの解説及び言語処理系の考察}
  \chapter{本実験全体の考察及び感想}
  \chapter{ソースコード}
\end{document}
